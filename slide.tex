%%=======================================================================
% !Mode:: "TeX:UTF-8"
% !TEX program  = XeLaTeX
%%=======================================================================
% 模板名称:zafu-beamer
% 模板版本:V0.1.0
% 模板作者:rogeryoungh
% 联系作者:rogeryoungh@foxmail
% 模板适用:浙江农林大学风格 Beamer 模板
% 模板编译:手动编译方法参看 README.md
% 更新时间:2023/05/18
% 模板帮助:推荐前往模板的 GitHub 仓库获取最新文件,地址:
%           https://github.com/rogeryoungh/zafu-beamer-theme
%%=======================================================================

% \documentclass{beamer}
% \usepackage{ctex, hyperref}

\documentclass{ctexbeamer}
% \documentclass[aspectratio=169]{ctexbeamer} % 设置长宽比为 16:9
\setCJKmainfont[BoldFont={Noto Serif CJK SC Bold},ItalicFont={Noto Sans CJK SC}]{Noto Serif CJK SC}
\setCJKsansfont{Noto Sans CJK SC}
\usefonttheme[onlymath]{serif}
\usepackage[T1]{fontenc}

% other packages
\usepackage{latexsym,amsmath,xcolor,multicol,booktabs,calligra}
\usepackage{graphicx,pstricks,listings,stackengine,hologo,hyperref}

\definecolor{linkcolor}{rgb}{0.5,0,0}
\definecolor{frenchplum}{RGB}{190,20,83}

\newcommand\mhref[2]{{\color{linkcolor}\href{#1}{#2}}}
\newcommand\murl[1]{{\color{linkcolor} \url{#1}}}

\author{\href{https://rogery.dev/}{rogeryoungh}}
\title{ZAFU Beamer Theme}
\subtitle{浙江农林大学风格 Beamer 模板}
\institute{浙江农林大学 数学与计算机学院}
\date{2023 年 5 月 18 日}
\usetheme{ZAFU}

% defs
\def\cmd#1{\texttt{\color{red}\footnotesize $\backslash$#1}}
\def\env#1{\texttt{\color{blue}\footnotesize #1}}
\definecolor{deepblue}{rgb}{0,0,0.5}
\definecolor{deepred}{rgb}{0.5,0,0}
\definecolor{deepgreen}{rgb}{0,0.5,0}
\definecolor{halfgray}{gray}{0.45}

\lstset{
	basicstyle=\small\linespread{1.0}\ttfamily \lst@ifdisplaystyle\fi,
	% Code
	columns=fullflexible,
	tabsize=3,
	showspaces=false,
	showstringspaces=false,
	showtabs=false,
	breaklines=true,
	numbers=left,
	numbersep=10pt,
	xleftmargin=2em,
	stepnumber=1,
	firstnumber=1,
	numberfirstline=true,
	% Code design
	keywordstyle=\bfseries,
	commentstyle=\color{codecolorcomments},
	numberstyle=\ttfamily,
	breakatwhitespace=false,
	breaklines=true,
	captionpos=t,
	keepspaces=true,
	%frame
	frame=tb,
	framerule=.5pt,
	framexleftmargin=7mm,
	rulecolor=\color{black},
	%backgroundcolor=\color{codecolorbg},
	abovecaptionskip=0pt,
	belowcaptionskip=5pt,
}

\begin{document}

% \kaishu
% \sffamily
\rmfamily
\begin{frame}
	\titlepage
	\begin{figure}[htpb]
		\begin{center}
			\includegraphics[width=0.175\linewidth]{pic/zafu-logo.png}
		\end{center}
	\end{figure}
\end{frame}

\begin{frame}
	\tableofcontents[sectionstyle=show,subsectionstyle=show/shaded/hide,subsubsectionstyle=show/shaded/hide]
\end{frame}


\section{在前面的话}

\begin{frame}{什么是 beamer}
	\begin{itemize}
		\item \LaTeX{} 是一种排版软件,由于公式支持较好,广泛的应用于数学、物理、计算机等方向的理工科学术界。
		\item Beamer 是 \LaTeX{} 的一种文档类,适合演讲,主要用于学术报告。
		\item Beamer 填充内容比较方便,因为你多半已经有了一份 \LaTeX{} 格式的论文源码,重新录入到 PPT 远比 beamer 麻烦。
	\end{itemize}
\end{frame}

\begin{frame}{为什么要使用 beamer}
	\begin{itemize}
		\item 学术性质的报告更加注重内容本身,而非花哨的动画,要求简洁直观。
		\item 用 Beamer 可以强制用户把内容提炼成要点以 itemize 的结构展示出来。
		\item 强制用户定义清晰的文章结构,有助于养成良好的论文写作习惯。
	\end{itemize}
\end{frame}


\begin{frame}{为什么要使用 beamer}
	\begin{itemize}
		\item \LaTeX 广泛用于学术界,期刊会议论文模板
	\end{itemize}
	\begin{table}[h]
		\centering
		\begin{tabular}{c|c}
			Microsoft\textsuperscript{\textregistered}  Word & \LaTeX        \\
			\hline
			文字处理工具                                           & 专业排版软件        \\
			容易上手,简单直观                                        & 容易上手          \\
			所见即所得                                            & 所见即所想,所想即所得   \\
			高级功能不易掌握                                         & 进阶难,但一般用不到    \\
			处理长文档需要丰富经验                                      & 和短文档处理基本无异    \\
			花费大量时间调格式                                        & 无需担心格式,专心作者内容 \\
			公式排版差强人意                                         & 尤其擅长公式排版      \\
			二进制格式,兼容性差                                       & 文本文件,易读、稳定    \\
			付费商业许可                                           & 自由免费使用        \\
		\end{tabular}
	\end{table}
\end{frame}

\begin{frame}{为什么要使用 beamer}
	\begin{itemize} % [<+-| alert@+>] % 当然,除了alert,手动在里面插 \pause 也行
		\item 因为专业原因我需要使用 \LaTeX{},也有做报告的需求。
		\item 许多学校都有自己的 beamer主题,我就抄来了一份。 \pause
		\item 请使用基于\hologo{XeLaTeX} 编译,支持中文。
		\item GitHub项目地址位于 \murl{https://github.com/rogeryoungh/zafu-beamer-theme},如果有 bug 或者功能请求可以去里面提 issue。
	\end{itemize}
\end{frame}

\begin{frame}{参考模板}
	\begin{itemize}
		\item 在魔改样式的过程中,参考了不少优秀模板,非常感谢。
		\item \mhref{https://github.com/inFaaa/PKU-Beamer-Theme}{PKU Beamer},inFaaa。
		\item \mhref{https://github.com/tuna/THU-Beamer-Theme}{THU Beamer},Trinkle23897。
		\item \mhref{https://github.com/JinLingxi/Cumtb-Beamer}{CUMTB Beamer},槿灵兮。
		\item \mhref{https://github.com/fuujiro/DLUT-Beamer-Slide-V2}{DLUT Beamer},fuujiro。
		\item 为了实现一校一Beamer而努力!\cite{origin}
	\end{itemize}
\end{frame}

\section{简单介绍}

\begin{frame}{这一份主题与源文件所给的原始的THU Beamer Theme区别在于}
	\begin{block}{说明}
		\begin{itemize}
			\item 我偏好中文衬线字体。
			\item 更多该模板的功能可以参考 \murl{https://www.latexstudio.net/archives/4051.html}
			\item 下面列举出了一些 beamer 的用法,部分节选自 \murl{https://tuna.moe/event/2018/latex/}
			\item 有关 \LaTeX{} 的学习可以查看 \murl{https://zhuanlan.zhihu.com/p/521649367}
		\end{itemize}
	\end{block}
\end{frame}

\subsection{排版举例}

\begin{frame}{排版举例}
	\begin{exampleblock}{无编号公式} % 加 * 
		\begin{equation*}
			J(\theta) = \mathbb{E}_{\pi_\theta}[G_t] = \sum_{s\in\mathcal{S}} d^\pi (s)V^\pi(s)=\sum_{s\in\mathcal{S}} d^\pi(s)\sum_{a\in\mathcal{A}}\pi_\theta(a|s)Q^\pi(s,a)
		\end{equation*}
	\end{exampleblock}
\end{frame}

\begin{frame}
	\begin{exampleblock}{多行多列公式\footnote{如果公式中有文字出现,请用 $\backslash$mathrm\{\} 或者 $\backslash$text\{\} 包含,不然就会变成 $clip$,在公式里看起来比 $\mathrm{clip}$ 丑非常多。}}
		% 使用 & 分隔
		\begin{align}
			Q_\mathrm{target} & =r+\gamma Q^\pi(s^\prime, \pi_\theta(s^\prime)+\epsilon)  \\
			\epsilon          & \sim\mathrm{clip}(\mathcal{N}(0, \sigma), -c, c)\nonumber
		\end{align}
	\end{exampleblock}
\end{frame}

\begin{frame}
	\begin{exampleblock}{编号多行公式}
		% Taken from Mathmode.tex
		\begin{multline}
			A=\lim_{n\rightarrow\infty}\Delta x\left(a^{2}+\left(a^{2}+2a\Delta x+\left(\Delta x\right)^{2}\right)\right.\label{eq:reset}\\
			+\left(a^{2}+2\cdot2a\Delta x+2^{2}\left(\Delta x\right)^{2}\right)\\
			+\left(a^{2}+2\cdot3a\Delta x+3^{2}\left(\Delta x\right)^{2}\right)\\
			+\ldots\\
			\left.+\left(a^{2}+2\cdot(n-1)a\Delta x+(n-1)^{2}\left(\Delta x\right)^{2}\right)\right)\\
			=\frac{1}{3}\left(b^{3}-a^{3}\right)
		\end{multline}
	\end{exampleblock}
\end{frame}

\begin{frame}{图形与分栏}
	% From thuthesis user guide.
	\begin{minipage}[c]{0.3\linewidth}
		\psset{unit=0.8cm}
		\begin{pspicture}(-1.75,-3)(3.25,4)
			\psline[linewidth=0.25pt](0,0)(0,4)
			\rput[tl]{0}(0.2,2){$\vec e_z$}
			\rput[tr]{0}(-0.9,1.4){$\vec e$}
			\rput[tl]{0}(2.8,-1.1){$\vec C_{ptm{ext}}$}
			\rput[br]{0}(-0.3,2.1){$\theta$}
			\rput{25}(0,0){%
				\psframe[fillstyle=solid,fillcolor=lightgray,linewidth=.8pt](-0.1,-3.2)(0.1,0)}
			\rput{25}(0,0){%
				\psellipse[fillstyle=solid,fillcolor=yellow,linewidth=3pt](0,0)(1.5,0.5)}
			\rput{25}(0,0){%
				\psframe[fillstyle=solid,fillcolor=lightgray,linewidth=.8pt](-0.1,0)(0.1,3.2)}
			\rput{25}(0,0){\psline[linecolor=red,linewidth=1.5pt]{->}(0,0)(0.,2)}
			%           \psRotation{0}(0,3.5){$\dot\phi$}
			%           \psRotation{25}(-1.2,2.6){$\dot\psi$}
			\psline[linecolor=red,linewidth=1.25pt]{->}(0,0)(0,2)
			\psline[linecolor=red,linewidth=1.25pt]{->}(0,0)(3,-1)
			\psline[linecolor=red,linewidth=1.25pt]{->}(0,0)(2.85,-0.95)
			\psarc{->}{2.1}{90}{112.5}
			\rput[bl](.1,.01){C}
		\end{pspicture}
	\end{minipage}\hspace{1cm}
	\begin{minipage}{0.5\linewidth}
		\medskip
		%\hspace{2cm}
		\begin{figure}[h]
			\centering
			\includegraphics[height=.5\textheight]{pic/zafu-logo.png}
		\end{figure}
	\end{minipage}
\end{frame}

\begin{frame}[fragile]{\LaTeX{} 常用命令}
	\begin{exampleblock}{命令}
		\centering
		\footnotesize
		\begin{tabular}{llll}
			\cmd{chapter}   & \cmd{section} & \cmd{subsection} & \cmd{paragraph}       \\
			章               & 节             & 小节               & 带题头段落                 \\\hline
			\cmd{centering} & \cmd{emph}    & \cmd{verb}       & \cmd{url}             \\
			居中对齐            & 强调            & 原样输出             & 超链接                   \\\hline
			\cmd{footnote}  & \cmd{item}    & \cmd{caption}    & \cmd{includegraphics} \\
			脚注              & 列表条目          & 标题               & 插入图片                  \\\hline
			\cmd{label}     & \cmd{cite}    & \cmd{ref}                                \\
			标号              & 引用参考文献        & 引用图表公式等                                  \\\hline
		\end{tabular}
	\end{exampleblock}
	\begin{exampleblock}{环境}
		\centering
		\footnotesize
		\begin{tabular}{lll}
			\env{table}   & \env{figure}    & \env{equation}    \\
			表格            & 图片              & 公式                \\\hline
			\env{itemize} & \env{enumerate} & \env{description} \\
			无编号列表         & 编号列表            & 描述                \\\hline
		\end{tabular}
	\end{exampleblock}
\end{frame}

\begin{frame}[fragile]{\LaTeX{} 环境命令举例}
	\begin{minipage}{0.5\linewidth}
		\begin{lstlisting}[language=TeX]
\begin{itemize}
  \item A \item B
  \item C
  \begin{itemize}
    \item C-1
  \end{itemize}
\end{itemize}
\end{lstlisting}
	\end{minipage}\hspace{1cm}
	\begin{minipage}{0.3\linewidth}
		\begin{itemize}
			\item A
			\item B
			\item C
			\begin{itemize}
				\item C-1
			\end{itemize}
		\end{itemize}
	\end{minipage}
	\medskip
\end{frame}


\begin{frame}[fragile]
	\begin{minipage}{0.5\linewidth}
		\begin{lstlisting}[language=TeX]
\begin{enumerate}
  \item 巨佬 \item 大佬
  \item 萌新
  \begin{itemize}
    \item 萌新
    \item[n+e] 瑟瑟发抖
  \end{itemize}
\end{enumerate}
\end{lstlisting}
	\end{minipage}\hspace{1cm}
	\begin{minipage}{0.3\linewidth}
		\begin{enumerate}
			\item 巨佬
			\item 大佬
			\item 萌新
			\begin{itemize}
				\item 萌新
				\item[n+e] 瑟瑟发抖
			\end{itemize}
		\end{enumerate}
	\end{minipage}
\end{frame}

\begin{frame}[fragile]{\LaTeX{} 数学公式}
	\begin{columns}
		\begin{column}{.55\textwidth}
			\begin{lstlisting}[language=TeX]
$V = \frac{4}{3}\pi r^3$

\[
  V = \frac{4}{3}\pi r^3
\]

\begin{equation}
  \label{eq:vsphere}
  V = \frac{4}{3}\pi r^3
\end{equation}
\end{lstlisting}
		\end{column}
		\begin{column}{.4\textwidth}
			$V = \frac{4}{3}\pi r^3$
			\[
				V = \frac{4}{3}\pi r^3
			\]
			\begin{equation}
				\label{eq:vsphere}
				V = \frac{4}{3}\pi r^3
			\end{equation}
		\end{column}
	\end{columns}
	\begin{itemize}
		\item 更多内容请看 \href{https://zh.wikipedia.org/wiki/Help:数学公式}{\color{purple}{这里}}
	\end{itemize}
\end{frame}

\begin{frame}[fragile]
	\begin{columns}
		\column{.6\textwidth}
		\begin{lstlisting}[language=TeX]
\begin{table}[htbp]
\caption{编号与含义}
\label{tab:number}
\centering
\begin{tabular}{cl}
    \toprule
    编号 & 含义 \\
    \midrule
    1 & 4.0 \\
    2 & 3.7 \\
    \bottomrule
\end{tabular}
\end{table}
公式~(\ref{eq:vsphere}) 的
编号与含义请参见
表~\ref{tab:number}。
\end{lstlisting}
		\column{.4\textwidth}
		\begin{table}[htpb]
			\centering
			\caption{编号与含义}
			\label{tab:number}
			\begin{tabular}{cl}\toprule
				编号 & 含义  \\\midrule
				1  & 4.0 \\
				2  & 3.7 \\\bottomrule
			\end{tabular}
		\end{table}
		\normalsize 公式~(\ref{eq:vsphere})的编号与含义请参见表~\ref{tab:number}。
	\end{columns}
\end{frame}

\begin{frame}{作图}
	\begin{itemize}
		\item 矢量图建议 PDF,基本上软件都能导出 PDF,一些远古教程还有 EPS、PS 等格式。 
		\item 标量图 png, jpg, tiff……提高清晰度,避免发虚。
	\end{itemize}
	\begin{figure}[htpb]
		\centering
		\includegraphics[width=0.2\linewidth]{pic/zafu-logo.png}
		\caption{这个校徽虽然不是矢量图,但还挺清楚。}
	\end{figure}
\end{frame}

\section{There Is No Largest Prime Number}

\begin{frame}
	\frametitle{There Is No Largest Prime Number}
	\framesubtitle{The proof uses \textit{reductio ad absurdum}.}
	\begin{theorem}\normalfont \rmfamily
	  There is no largest prime number.
	\end{theorem} \pause
	\begin{proof}
	  \begin{enumerate}
		\item<2->Suppose $p$ were the largest prime number.
		\item<3->Let $q$ be the product of the first $p$ numbers.
		\item<4->Then $q + 1$ is not divisible by any of them.
		\item<5->But $q + 1$ is greater than $1$, thus divisible by some prime number not in the first $p$ numbers.\qedhere
	  \end{enumerate}
	\end{proof}
	\uncover<5->{The proof used \textit{reductio ad absurdum}.}
\end{frame}

\section{参考文献}
\begin{frame}[allowframebreaks]
	\bibliography{ref}
	\bibliographystyle{alpha}
	% 如果参考文献太多的话,可以像下面这样调整字体:
	% \tiny\bibliographystyle{alpha}
\end{frame}

\begin{frame}{结语}
	\begin{center}
		{\Huge\calligra Thanks!}
	\end{center}
\end{frame}

\end{document}